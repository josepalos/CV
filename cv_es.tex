\documentclass{cv}
\usepackage[spanish]{babel}

\begin{document}
\makecvtitle

\section{Objetivo profesional}
Actualmente mi objetivo profesional es ampliar mis conocimientos
sobre el mundo de la inteligencia artificial y los algoritmos de optimización
de problemas, para dedicarme en un futuro a la investigación en este campo.

\section{Educación}
\cventry{2011--2013}
	{Estudios de bachillerato tecnológico}
	{Institut Ciutat de Balaguer}
	{Balaguer (Lleida)}
	{}
	{}
\cventry{2014--2019}
	{Doble grado en Ingeniería Informática y Administración y Dirección de Empresas}
	{Universitat de Lleida}
	{Lleida}
	{Mención al mejor expediente de la promoción}
	{}
\cventry{2016--2017}
	{[ERASMUS] Doble grado en Ingeniería Informática y Administración y Dirección de Empresas}
	{Vaasan ammattikorkeakoulu (VAMK)}
	{Vaasa (Finlándia)}
	{}
	{}
\cventry{2019--2020}
	{Máster en ciencia de datos}
	{Universitat Oberta de Catalunya}
	{(Online)}
	{}
	{}

\section{Formación complementaria}
\cventry{2012}
	{\textbf{Certificado ACTIC} (Acreditación de Competencias en Tecnologías de
		la Información y la Comunicación)}
	{Generalitat de Catalunya}
	{Balaguer (Lleida)}
	{\textbf{Nivel 2} (medio)}
	{}
\cventry{2014}
	{Curso de formación sobre \textbf{mercados financieros y bolsa}}
	{GAESCO y Universitat de Lleida}
	{Lleida}
	{}
	{}
	{}

\section{Experiencia}
\cventry{Julio i septiembre 2015}
	{Becario, colaboración en el proceso de matriculación para nuevos estudiantes}
	{Universitat de Lleida}
	{Lleida}
	{}
	{}
\cventry{Navidades 2015}
	{Monitor de robótica con Lego Mindstorms en el casal de navidad de Engijocs}
	{Engijocs}
	{Lleida}
	{}
	{}
\cventry{Febrero 2016--Abril 2016}
	{Becario, Responsable de la página web de la ``Fira del Treball''}
	{Universitat de Lleida}
	{Lleida}
	{}
	{\begin{itemize}
		\item Gestión de la página web de la feria mediante un CRM
		\item Atención a los asistentes de la feria para proporcionar información y resolver dudas
	\end{itemize}}
\cventry{Septiembre 2017--Diciembre 2018}
	{Ingeniero informático}
	{Invelon Technologies}
	{Lleida}
	{}
	{\begin{itemize}
		\item Desarrollo de un dispositivo IOT para gestionar impresoras 3D mediante
			comunicación \textit{serial} (tecnologías usadas: Python).
		\item Desarrollo del \textit{backend} para conectar la página web de la herramienta
			con el dispositivo IOT (tecnologías usadas: Python, Django).
	\end{itemize}}
\cventry{Noviembre 2018--Actualidad}
	{Becario, colaboración en proyectos de optimización}
	{Grup de recerca d'optimització, Universitat de Lleida}
	{Lleida}
	{}
	{\begin{itemize}
		\item Desarrollo y mantenimiento de un servicio SaaS para ejecutar
			y gestionar tareas en un cluster de alto rendimiento (tecnologías
			usadas: Python, Django, Websockets, Bash, Sun Grid Engine,
			herramientas de configuración automática (AC))
		\item Experimentación sobre el rendimiento de varios programas de
			resolución de programación lineal (ILP) mediante AC.
		\item Desarrollo de un proceso completo de AC de problemas de optimización
			codificados con ILP para datos reales de una empresa cliente
			(tecnologías usadas: Python, Bash, \textit{solvers} de AC y de ILP)
		\item Desarrollo de un paquete de programario para optimizar árboles de
			decisión (tecnologías usadas: Python, Bash, \textit{solvers} de SAT/MaxSAT)
	\end{itemize}}
\cventry{Febrero 2020--Julio 2020}
	{Profesor universitario}
	{ENTI Escola de Noves Tecnologies Interactives}
	{Barcelona}
	{}
	{Responsable de las asignaturas de \textit{Mecànica} y \textit{Informàtica gràfica}}
\cventry{Septiembre 2020--Enero 2021}
	{Profesor universitario}
	{Universitat de Lleida}
	{Lleida}
	{}
	{Profesor adjunto de la asignatura \textit{Intel·ligència Artificial}}
\cventry{Noviembre 2020--Actualidad}
	{Proyecto CAINALCO}
	{Universitat de Lleida}
	{Lleida}
	{}
	{Integración y aplicación de algoritmos AC para calibración inteligente
	de modelos epidemiológicos en el contexto de la COVID19}
\cventry{Febrero 2021--Julio 2021}
	{Profesor universitario}
	{ENTI Escola de Noves Tecnologies Interactives}
	{Barcelona}
	{}
	{Responsable de la asignatura de \textit{Mecànica}}

\section{Idiomas}
\cvlanguage{Catalán}
	{Lengua materna}
	{}
\cvlanguage{Castellano}
	{Lengua materna}
	{}
\cvlanguage{Inglés}
	{Nivel B2}
	{}

\subsection{Títulos}
\cventry{2013}
	{Inglés -- Certificado de nivel intermedio}
	{Escola Oficial d'Idiomes}
	{}
	{APTO}
	{}
\cventry{2013}
	{Inglés -- Edexcel Level I Certificate in ESOL Internacional
		\textbf{(CEF B2)}}
	{Pearson Education Ltd.}
	{}
	{PASS (WITH MERIT)}
	{}


\section{Competencias informáticas}
\subsection{Sistemas operativos}
\languageknowledge{Linux}{5}
\languageknowledge{Windows}{4}

\subsection{Herramientas}
\languageknowledge{Git}{5}
\languageknowledge{Bash/Shell}{4}
\languageknowledge{Ansible}{2}
\languageknowledge{Latex}{3}
\languageknowledge{Docker}{4}
\languageknowledge{Docker-compose}{4}
\languageknowledge{SQL}{4}
\languageknowledge{Arduino}{3}

\subsection{Lenguages de programación}
\languageknowledge{C}{3}
\languageknowledge{C++}{3}
\languageknowledge{Python}{5}
\languageknowledge{Java}{4}
\languageknowledge{R}{2}

\subsection{Tecnologías web}
\languageknowledge{HTML}{5}
\languageknowledge{CSS}{2}
\languageknowledge{PHP}{3}
\languageknowledge{Javascript}{3}
\languageknowledge{Django}{5}
\languageknowledge{Flask}{3}
\languageknowledge{Node}{2}
\languageknowledge{React}{2}
\languageknowledge{Express}{2}

\section{Otros}
\cventry{2014}
	{Concurso de simulación bursátil}
	{GAESCO i Universitat de Lleida}
	{Lleida}
	{Tercera posición}
	{}
\cventry{2014,15,18,19}
	{Voluntario como árbitro en la competición de robótica de la
		\textit{First Lego League}}
	{Universitat de Lleida}
	{Lleida}
	{}
	{}
\cventry{2014--Actualidad}
	{Voluntario de la \textit{Associació de Donants de Sang La Noguera}}
	{Donants de Sang La Noguera}
	{Balaguer}
	{}
	{}
\cventry{2017}
	{Participante en la HackEPS}
	{Universitat de Lleida}
	{Lleida}
	{Ganadores del reto BonÀrea}
	{} % TODO explain the project done
\cventry{2019}
	{Participante en la HackUPC}
	{Facultat d'Informàtica de Barcelona}
	{Barcelona}
	{}
	{} % TODO explain the project done
\cventry{2019}
	{Participante en la HackEPS}
	{Universitat de Lleida}
	{Lleida}
	{}
	{} % TODO explain the project done
\cventry{2019}
	{Premio AETI 2018/19 al mejor proyecto informático del grado}
	{Universitat de Lleida}
	{Lleida}
	{}
	{} % TODO add the project title

\end{document}
