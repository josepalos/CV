\documentclass[a4paper,12pt,final]{moderncv}
\usepackage[T1]{fontenc}
\usepackage[utf8]{inputenc}
\moderncvtheme{classic}

\firstname{Josep}
\familyname{Alòs Pascual}
\address{C\textbackslash La Plana 90}{25600 Balaguer, Lleida (Spain)}
\mobile{+34 685 20 10 27}
\email{josep.a.95@gmail.com}
\social[github]{josepalos}
%\photo[size pt]{name}
%\extrainfo{}
%\social[linkedin]{https://www.linkedin.com/in/josep-alòs-pascual-554358119/}

\begin{document}
\makecvtitle

\section{Objectiu professional}
Actualment el meu objectiu professional és ampliar el meu coneixement
sobre el món de la intel·ligència artificial i algoritmes d'optimització
de problemes, de cara a dedicar-me en un futur a la recerca en aquests camps.

\section{Educació}
\cventry{2011--2013}
	{Estudis de batxillerat tecnològic}
	{Institut Ciutat de Balaguer}
	{Balaguer (Lleida)}
	{}
	{}
\cventry{2014--2019}
	{Doble grau en Enginyeria Informàtica i Administració i Direcció
		d'Empreses}
	{Universitat de Lleida}
	{Lleida}
	{Menció a millor expedient de la promoció}
	{}
\cventry{2016--2017}
	{[ERASMUS] Doble grau en Enginyeria Informàtica i
		Administració i Direcció d'Empreses}
	{Vaasan ammattikorkeakoulu (VAMK)}
	{Vaasa (Finlàndia)}
	{}
	{}
\cventry{2019--actualitat}
	{Màster en ciència de dades}
	{Universitat Oberta de Catalunya}
	{(Online)}
	{}
	{}

\section{Formació complementaria}
\cventry{2012}
	{\textbf{Certificat ACTIC} (Acreditació de Competències en Tecnologíes de
		la Informació i la Comunicació)}
	{Generalitat de Catalunya}
	{Balaguer (Lleida)}
	{\textbf{Nivell 2} (mig)}
	{}
\cventry{2014}
	{Curs de formació sobre \textbf{mercats financers i borsa}}
	{GAESCO i Universitat de Lleida}
	{Lleida}
	{}
	{}
	{}

\section{Experiència}
\cventry{Juliol i setembre 2014}
	{Becari, colaboració en el procés de matricula per estudiants nous}
	{Universitat de Lleida}
	{Lleida}
	{}
	{}
\cventry{Nadal 2015}
	{Monitor de robòtica amb Lego Mindstorms al casal del nadal d'Engijocs}
	{Engijocs}
	{Lleida}
	{}
	{}
\cventry{Febrer 2016--Abril 2016}
	{Becari, Responsable de la pàgina web de la ``Fira del Treball''}
	{Universitat de Lleida}
	{Lleida}
	{}
	{}
\cventry{Setembre 2017--Desembre 2018}
	{Enginyer informàtic}
	{Invelon Technologies}
	{Lleida}
	{}
	{}
\cventry{Novembre 2018--Actualitat}
	{Becari, colaboració en projectes d'optimització}
	{Grup de recerca d'optimització, Universitat de Lleida}
	{Lleida}
	{}
	{}

\section{Llengues}
\cvlanguage{Català}
	{Llengua materna}
	{}
\cvlanguage{Castellà}
	{Llengua materna}
	{}
\cvlanguage{Anglès}
	{Nivell B2}
	{}

\subsection{Títols}
\cventry{2013}
	{ANGLÈS -- Certificat de nivell intermedi}
	{Escola Oficial d'Idiomes}
	{}
	{APTE}
	{}
\cventry{2013}
	{ANGLÈS -- Edexcel Level I Certificate in ESOL Internacional
		\textbf{(CEF B2)}}
	{Pearson Education Ltd.}
	{}
	{PASS (WITH MERIT)}
	{}


\section{Competències informàtiques}
\cvcomputer{Sistemes operatius}
		{Linux, Windows}
	{Eines}
		{Git, Shell, Ansible, Latex}
\cvcomputer{Llenguatges de programació}
		{C, C++, Python, Java, R, bash,}
	{Tecnologies Web}
		{HTML, Javascript, PHP, Django, Flask}

\section{Altres}
\cventry{2014}
	{Concurs de simulació bursàtil}
	{GAESCO i Universitat de Lleida}
	{Lleida}
	{Tercera posició}
	{}
\cventry{2014,15,18,19}
	{Voluntari com a àrbitre en la competició de robòtica de la
		\textit{First Lego League}}
	{Universitat de Lleida}
	{Lleida}
	{}
	{}
\cventry{2014--actualitat}
	{Voluntari de l'Associació de Donants de Sang La Noguera}
	{Donants de Sang La Noguera}
	{Balaguer}
	{}
	{}
\cventry{2017}
	{Participant en la HackEPS}
	{Universitat de Lleida}
	{Lleida}
	{Guanyadors del repte BonÀrea}
	{}
\cventry{2019}
	{Participant en la HackUPC}
	{Facultat d'Informàtica de Barcelona}
	{Barcelona}
	{}
	{}
\cventry{2019}
	{Participant en la HackEPS}
	{Universitat de Lleida}
	{Lleida}
	{}
	{}

\end{document}
