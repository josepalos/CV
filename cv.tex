\documentclass[a4paper,12pt,final]{moderncv}
\usepackage[T1]{fontenc}
\usepackage[utf8]{inputenc}
\usepackage[catalan]{babel}
\moderncvtheme{classic}

\colorlet{languagecolor}{red}
\colorlet{nolanguagecolor}{gray}
\newcount\languagecount
\newcommand\languageknowledge[2]
{%
	\hbox
	{%
		\makebox[4cm][l]{#1}%
		\languagecount=0
		\loop\ifnum\languagecount<#2
			\advance\languagecount1
			\textcolor{languagecolor}{$\bullet$}%
		\repeat
		\loop\ifnum\languagecount<5
			\advance\languagecount1
			\textcolor{nolanguagecolor}{$\bullet$}%
		\repeat
	}%
}

%\newcommand\languageknowledge[2]
%{%
%	\cventry{#1}{%
%		\languagecount=0
%		\loop\ifnum\languagecount<#2
%			\advance\languagecount1
%			\textcolor{languagecolor}{$\bullet$}%
%		\repeat
%		\loop\ifnum\languagecount<5
%			\advance\languagecount1
%			\textcolor{nolanguagecolor}{$\bullet$}%
%		\repeat
%	}%
%}

\firstname{Josep}
\familyname{Alòs Pascual}
\address{C\textbackslash La Plana 90}{25600 Balaguer, Lleida (Spain)}
\mobile{+34 685 20 10 27}
\email{josep.a.95@gmail.com}
\social[github]{josepalos}
%\photo[size pt]{name}
%\extrainfo{}
%\social[linkedin]{https://www.linkedin.com/in/josep-alòs-pascual-554358119/}

\begin{document}
\makecvtitle

\section{Objectiu professional}
Actualment el meu objectiu professional és ampliar el meu coneixement
sobre el món de la inte\lgem igència artificial i algoritmes d'optimització
de problemes, de cara a dedicar-me en un futur a la recerca en aquests camps.

\section{Educació}
\cventry{2011--2013}
	{Estudis de batxillerat tecnològic}
	{Institut Ciutat de Balaguer}
	{Balaguer (Lleida)}
	{}
	{}
\cventry{2014--2019}
	{Doble grau en Enginyeria Informàtica i Administració i Direcció
		d'Empreses}
	{Universitat de Lleida}
	{Lleida}
	{Menció a millor expedient de la promoció}
	{}
\cventry{2016--2017}
	{[ERASMUS] Doble grau en Enginyeria Informàtica i
		Administració i Direcció d'Empreses}
	{Vaasan ammattikorkeakoulu (VAMK)}
	{Vaasa (Finlàndia)}
	{}
	{}
\cventry{2019--2020}
	{Màster en ciència de dades}
	{Universitat Oberta de Catalunya}
	{(Online)}
	{}
	{}

\section{Formació complementaria}
\cventry{2012}
	{\textbf{Certificat ACTIC} (Acreditació de Competències en Tecnologíes de
		la Informació i la Comunicació)}
	{Generalitat de Catalunya}
	{Balaguer (Lleida)}
	{\textbf{Nivell 2} (mig)}
	{}
\cventry{2014}
	{Curs de formació sobre \textbf{mercats financers i borsa}}
	{GAESCO i Universitat de Lleida}
	{Lleida}
	{}
	{}
	{}

\section{Experiència}
\cventry{Juliol i setembre 2015}
	{Becari, colaboració en el procés de matricula per estudiants nous}
	{Universitat de Lleida}
	{Lleida}
	{}
	{}
\cventry{Nadal 2015}
	{Monitor de robòtica amb Lego Mindstorms al casal del nadal d'Engijocs}
	{Engijocs}
	{Lleida}
	{}
	{}
\cventry{Febrer 2016--Abril 2016}
	{Becari, Responsable de la pàgina web de la ``Fira del Treball''}
	{Universitat de Lleida}
	{Lleida}
	{}
	{\begin{itemize}
		\item Gestió de la pàgina web de la fira mitjançant un CRM
		\item Atenció als assistents de la fira per proporcionar informació i resoldre dubtes.
	\end{itemize}}
\cventry{Setembre 2017--Desembre 2018}
	{Enginyer informàtic}
	{Invelon Technologies}
	{Lleida}
	{}
	{\begin{itemize}
		\item Desenvolupament d'un dispositiu IOT per gestionar impressores 3D
			mitjançant communicació \textit{serial} (tecnologies utilitzades: Python).
		\item Desenvolupament del \textit{backend} per connectar la pàgina web
			de l'eina amb el dispositiu IOT (tecnologies utilitzades: Python, Django).
	\end{itemize}}
\cventry{Novembre 2018--Actualitat}
	{Becari, colaboració en projectes d'optimització}
	{Grup de recerca d'optimització, Universitat de Lleida}
	{Lleida}
	{}
	{\begin{itemize}
		\item Desenvolupament i manteniment d'un servei SaaS per executar i
			gestionar tasques en un clúster d'alt rendiment (tecnologies
			utilitzades: Python, Django, Websockets, Bash, Sun Grid Engine,
			eines de configuració automàtica (AC))
		\item Experimentació sobre el rendiment de diferents programes de
			resolució de programació lineal (ILP) mitjançant AC.
		\item Desenvolupament d'un procés complet d'AC de problemes
			d'optimització mitjançant ILP per un procés d'una empresa client
			(tecnologies utilitzades: Python, Bash, \textit{solvers} d'AC i 
			ILP)
		\item Desenvolupament d'un paquet de programari per optimitzar
			arbres de decisió (tecnologies utilitzades: Python, Bash,
			\textit{solvers} de SAT/MaxSAT)
	\end{itemize}}
\cventry{Febrer 2020--Juliol 2020}
	{Professor universitari}
	{ENTI Escola de Noves Tecnologies Interactives}
	{Barcelona}
	{}
	{Responsable de les assignatures de Mecànica i d'Informàtica gràfica}
\cventry{Setembre 2020--Gener 2021}
	{Professor universitari}
	{Universitat de Lleida}
	{Lleida}
	{}
	{Professor adscrit de l'assignatura Intel·ligència Artificial}
\cventry{}
	{Projecte CAINALCO}
	{Universitat de Lleida}
	{Lleida}
	{}
	{Integració i aplicació d'algoritmes de configuració automàtica per
	calibració intel·ligent de models epidemiològics en el context de
	la COVID19}
\cventry{Febrer 2021--Juliol 2021}
	{Professor universitari}
	{ENTI Escola de Noves Tecnologies Interactives}
	{Barcelona}
	{}
	{Responsable de l'assignatura de Mecànica}

\section{Llengues}
\cvlanguage{Català}
	{Llengua materna}
	{}
\cvlanguage{Castellà}
	{Llengua materna}
	{}
\cvlanguage{Anglès}
	{Nivell B2}
	{}

\subsection{Títols}
\cventry{2013}
	{ANGLÈS -- Certificat de nivell intermedi}
	{Escola Oficial d'Idiomes}
	{}
	{APTE}
	{}
\cventry{2013}
	{ANGLÈS -- Edexcel Level I Certificate in ESOL Internacional
		\textbf{(CEF B2)}}
	{Pearson Education Ltd.}
	{}
	{PASS (WITH MERIT)}
	{}


\section{Competències informàtiques}
\subsection{Sistemes operatius}
\languageknowledge{Linux}{5}
\languageknowledge{Windows}{4}

\subsection{Eines}
\languageknowledge{Git}{5}
\languageknowledge{Bash/Shell}{4}
\languageknowledge{Ansible}{2}
\languageknowledge{Latex}{3}
\languageknowledge{Docker}{4}
\languageknowledge{Docker-compose}{4}
\languageknowledge{SQL}{4}
\languageknowledge{Arduino}{3}

\subsection{Llenguatges de programació}
\languageknowledge{C}{3}
\languageknowledge{C++}{3}
\languageknowledge{Python}{5}
\languageknowledge{Java}{4}
\languageknowledge{R}{2}

\subsection{Tecnologies web}
\languageknowledge{HTML}{5}
\languageknowledge{CSS}{2}
\languageknowledge{PHP}{3}
\languageknowledge{Javascript}{3}
\languageknowledge{Django}{5}
\languageknowledge{Flask}{3}
\languageknowledge{Node}{2}
\languageknowledge{React}{2}
\languageknowledge{Express}{2}

\section{Altres}
\cventry{2014}
	{Concurs de simulació bursàtil}
	{GAESCO i Universitat de Lleida}
	{Lleida}
	{Tercera posició}
	{}
\cventry{2014,15,18,19}
	{Voluntari com a àrbitre en la competició de robòtica de la
		\textit{First Lego League}}
	{Universitat de Lleida}
	{Lleida}
	{}
	{}
\cventry{2014--actualitat}
	{Voluntari de l'Associació de Donants de Sang La Noguera}
	{Donants de Sang La Noguera}
	{Balaguer}
	{}
	{}
\cventry{2017}
	{Participant en la HackEPS}
	{Universitat de Lleida}
	{Lleida}
	{Guanyadors del repte BonÀrea}
	{} % TODO explain the project done
\cventry{2019}
	{Participant en la HackUPC}
	{Facultat d'Informàtica de Barcelona}
	{Barcelona}
	{}
	{} % TODO explain the project done
\cventry{2019}
	{Participant en la HackEPS}
	{Universitat de Lleida}
	{Lleida}
	{}
	{} % TODO explain the project done
\cventry{2019}
	{Premi AETI 2018/19 al millor projecte informàtic de grau}
	{Universitat de Lleida}
	{Lleida}
	{}
	{} % TODO add the project title

\end{document}
